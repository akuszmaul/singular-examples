\documentclass[10pt]{article}
\usepackage[margin=1in]{geometry}
\usepackage{minted}
\title{Matrix Multiplication on Singular S1 --- An Example}
\begin{document}
\maketitle

\section{Outline}
Below is the basic outline of the matrix multiplication code, with comments outlining what each function will do.  We will then look at sections of the complete code in depth, starting with the main function.

\begin{minted}{c}
/*********************************************************************************
    (c) Copyright 2011-2016 Singular Computing LLC

    This file and related materials are Confidential Information
    and Proprietary Property of Singular Computing LLC.

**********************************************************************************/


/*

Simple matrix multiplication example code for S1 using Nova.

To compile: 
gcc -lm -o matrixMultiplication matrixMultiplication.c scNova.c scAcceleratorAPI.c scEmulator.c scArithmetic178.c pmbus.c


 We keep a matrix (A) stored persistently in the S1,
 and multiply it with a temporary matrix (B), storing the result in A.

  For this code, A and B have fixed, identical, square NxN shapes, where N=8.
  Matrix elements are stored one per core.

*/


#include <stdio.h>
#include <stdlib.h>
#include <string.h>
#include <math.h>
#include <sys/types.h>
#include <sys/stat.h>
#include <time.h>

// Include the Singular Software libraries.
#include "scAcceleratorAPI.h"
#include "scNova.h"


int emulated;                     // 1 for emulated machine.  0 for real S1 hardware.

// If emulated, scTotalCyclesTake is the total number of cycles taken to run the last kernel.
extern int scTotalCyclesTaken;

// If emulated, *scEmulatedMachineState is a pointer to the emulated machine's state.
extern MachineState *scEmulatedMachineState;  

extern LLKernel *llKernel;

// Variables for the size info of the machine.
int chipRows;
int chipCols;
int apeRows;
int apeCols;

int traceFlags;

// Define the length of the square matrices.
#define N 8

// Declare space for matrices A and B on the CPU, in float format.
float floatA[N][N];
float floatB[N][N];

// Declare space for a matrix on the CPU, in approx format (16 bits), aligned at 64 bits.
uint64_t approxM[(N*N)/4];

// Declare often used Nova Constants.
Declare(a0);
Declare(a1);

// Declare names of A and B matrices in Ape memory.
Declare(A);
Declare(B);

void defineNames () {
// Initialization routine to define the names above.
a0 = AConst(0);
a1 = AConst(1);
ApeMem(A, Approx);
ApeMem(B, Approx);
}

void emitCopyMatrixFromCUToApes(int cuAddress, int apeAddress) {
    // Copy N*N 16 bit data words
    // from CU Data Memory starting at cuAddress to the Ape grid.
}

void emitCopyMatrixFromApesToCU(int apeAddress, int cuAddress) {
    // Copy N*N 16 bit data words
    // from the Ape grid to CU Data Memory starting at cuAddress.
}

void copyBToCU () {
    // Convert floatB to approx and copy to matrix B in CU Data Memory
}

void copyAFromCU () {
    // Copy matrix A in CU Data Memory to floatA in the CPU (and convert from approx to
// float).

}

void emitMatrixSet () {
    // Make matrix A equal to matrix B.
Set(A, B);
}


void emitGetTorus(scExpr x, int dir){
    // This function is similar to getApe() in Nova.h.  However, getApe() is not set up for
    // a torus configuration, and this function is.
}

void emitMatrixMul () {
    // Emit code for matrix multiply:  A = A * B.
    // Steps:
    // 1) Order the apes by row and column number.
    // 2) Do the initial shifting of the rows of matrix A and columns of matrix B.
    // 3) Multiply the A and B element in each Ape, save the result, shift the matrices, and
// repeat until the matrices are finished multiplying.
}


void check (char *testname, int i, int j, float expected) {
    // Checks if floatA[i][j] is close to expected, and prints an error otherwise.
}


void tests () {
    // Runs the matrix multiplication and check that it works.
    // We do this by emitting instructions into a kernel and then executing the kernel.
    // Steps:
    // 1) Kernel instruction: Copy B from CU to Apes.
// 2) Kernel instruction: Sets A = B
// 3) Kernel instruction: Run the matrix multiplication.  A = A * B.
// 4) Kernel instruction: Copy A from Apes to CU, and wait for CU to say to continue.
    // 5) Check to see if the matrix multiplication is correct.
    // 6) Execute the kernel.


}
int main (int argc, char *argv[]) {
    // Process the command line arguments, create a machine, and run the
    // tests for the matrix multiplication.
}
\end{minted}

This matrix multiplication code will not run on the real Singular Computing machine, because it’s coded for a torus neural network, which the real machine is not.  However, later on we will code the real machine to work as a torus in such a way that the programmer does not need to know that the real machine’s hardware isn’t set up that way.  So, later this code may be able to be minimally modified and it will run on the real machine. \par  
In order to run this code, there are three command line arguments: \par
\begin{enumerate}
\item The name of the file to be compiled.  
\item Whether the code should be run on the real machine or the emulated machine. \item What trace flags to show.
\end{enumerate}
The command should look like, “./matrixMultiplication emulated 0”.  For now it’s always going to be emulated, but the trace flags used can be changed to give you different information about what’s happening within the machine as the code runs. \par
    First, let’s look at the main function:

\inputminted{c}{mm-main.c}

Here I should put some more info about each Trace Flag option. \par

Next, let’s look at emitCopyMatrixFromCUToApes(int cuAddress, int apeAddress):

\inputminted{c}{mm-emitCopyMatrixFromCUToApes.c}

Next, let’s look at emitCopyMatrixFromApesToCU(int apeAddress, int cuAddress), which is very similar to the previous function in reverse.

\inputminted{c}{mm-emitCopyMatrixFromApesToCU.c}

Next, let’s look at the code that copies matrix B from the CPU into the CU: \par

\inputminted{c}{mm-copyBToCU.c}

In order to copy matrix A from the CU into the CPU, we must use a similar function to copyBToCU.  Next, let’s look at copyAFromCU: \par

\inputminted{c}{mm-copyAFromCU.c}

We’re getting close to the actual matrix multiplication function.  Let’s just take a quick glance at the emitGetTorus function first.  This function allows us to specify an ape variable (such as matrix A) and a direction (such as South).  In that example, every ape is directed to replace their own matrix A value with the value of matrix A from the ape South of them.  The bottom row of apes will circularly take the value from the top row.  There is an apeGet function in that works almost the same, only it does not allow for a torus configuration.  If apeGet is used and given the direction South, the bottom row of apes will all reach off the grid and come back with zeros, instead of the values in the top row.

\inputminted{c}{mm-emitGetTorus.c}

Now let’s look at the matrix multiplication function.  First, we have to understand how matrix multiplication works with a network.  The following algorithm is gotten from Cypher and Sanz Chapter 5.6. \par
We want to multiply each row value of matrix A with the corresponding column value of matrix B.  Example: \par

\begin{tabular}{l*{13}{c}r}
  \\
  \\
  & & & Matrix A & & & & & & & & Matrix B & \\
  \\
  & &
  \textit{Column 0} &
  \textit{Column 1} &
  \textit{Column 2} & & & & & &
  \textit{Column 0} &
  \textit{Column 1} &
  \textit{Column 2}\\
  \\
  \textit{Row 0} & & A & B & C & & & & & & J & K & L \\
  \\
  \textit{Row 1} & & D & E & F & & & & & & M & N & O \\
  \\
  \textit{Row 2} & & G & H & I & & & & & & P & Q & R\\
  \\
  \\
\end{tabular}

Matrix A * Matrix B = Matrix C

\begin{tabular}{l*{26}{c}r}
  \\
  \\
  & & & & & & & & Matrix C\\
  \\
  & & & & &
  \textit{Column 0} & & &
  \textit{Column 1} & & &
  \textit{Column 2} \\
  \\
  \textit{Row 0} & & & & &
  AJ + BM + CP &&& AK + BN + CQ &&& AL + BO + CR \\
  \\
  \textit{Row 1} & & & & &
  DJ + EM + FP &&& DK + EN + FQ &&& DL + EO + FR \\
  \\
  \textit{Row 2} & & & & &
  GJ + HM + IP &&& GK + HN + IQ &&& GL + HO + IR \\
  \\
  \\
\end{tabular}

Looking at C[0][0], we can see that AJ + BM + CP multiplies the values of row 0 in matrix A with the values of column 0 in matrix B.  For every value in C[i][j], the values in row i of matrix A are multiplied with the values of column j in matrix B.  In order to do matrix multiplication for, we can shift matrices A and B so that the corresponding values are held in the same ape. \par
First, we must circularly shift each row i of matrix A to the left i times. \par
  
\begin{tabular}{l*{9}{c}r}
  \\
  \\
  & & & Matrix A \\
  \\
  & &
  \textit{Column 0} &
  \textit{Column 1} &
  \textit{Column 2} & \\
  \\
  \textit{Row 0} & & A & B & C & // Row 0 shifts 0 times to the left, circularly.\\
  \\
  \textit{Row 1} & & E & F & D & // Row 1 shifts 1 times to the left, circularly.\\
  \\
  \textit{Row 2} & & I & G & H & // Row 2 shifts 2 times to the left, circularly.\\
  \\
  \\
\end{tabular}

Second, we must circularly shift each column j of matrix B upwards j times. \par

\begin{tabular}{l*{9}{c}r}
  \\
  \\
  & & & Matrix B \\
  \\
  & &
  \textit{Column 0} &
  \textit{Column 1} &
  \textit{Column 2} & \\
  \\
  \textit{Row 0} & & J & N & R & // Column 0 shifts upwards 0 times, circularly.\\
  \\
  \textit{Row 1} & & M & Q & L & // Column 1 shifts upwards 1 times, circularly.\\
  \\
  \textit{Row 2} & & P & K & O & // Column 2 shifts upwards 2 times, circularly.\\
  \\
  \\
\end{tabular}

At this point, each ape holds a pair of values from matrix A and matrix B that will be multiplied together to create matrix C.  Below are the bolded pairs that are currently held by each ape. \par

\begin{tabular}{l*{11}{c}r}
  \\
  \\
  && && Matrix C \\
  \\
  & &
  \textit{Column 0} &&
  \textit{Column 1} &&
  \textit{Column 2} \\
  \\
  \textit{Row 0} && \textbf{AJ} + BM + CP && AK + \textbf{BN} + CQ && AL + BO + \textbf{CR} \\
  \\
  \textit{Row 1} && DJ + \textbf{EM} + FP && DK + EN + \textbf{FQ} && \textbf{DL} + EO + FR \\
  \\
  \textit{Row 2} && GJ + HM + \textbf{IP} && \textbf{GK} + HN + IQ && GL + \textbf{HO} + IR \\
  \\
  \\
\end{tabular}

Each ape must now multiply their value of matrix A and their value of matrix B and then store the result in what will become matrix C. \par
The values in matrix A are now circularly shifted left once and the values in matrix B are circularly shifted upwards once.  This will, again, make it so that every ape holds a pair of corresponding matrix A and matrix B values that must be multiplied together. \par

\begin{tabular}{l*{9}{c}r}
  \\
  \\
  & & & Matrix A \\
  \\
  & &
  \textit{Column 0} &
  \textit{Column 1} &
  \textit{Column 2} & \\
  \\
  \textit{Row 0} & & B & C & A && // Shift each value to the left once (circularly).\\
  \\
  \textit{Row 1} & & F & D & E \\
  \\
  \textit{Row 2} & & G & H & I \\
  \\
  \\
\end{tabular}

\begin{tabular}{l*{9}{c}r}
  \\
  \\
  & & & Matrix B \\
  \\
  & &
  \textit{Column 0} &
  \textit{Column 1} &
  \textit{Column 2} & \\
  \\
  \textit{Row 0} & & M & Q & L && // Shift each value upwards once (circularly).\\
  \\
  \textit{Row 1} & & P & K & O \\
  \\
  \textit{Row 2} & & J & N & R \\
  \\
  \\
\end{tabular}

Below are the bolded pairs that are now held by each ape.  They haven't yet been multiplied together and put into matrix C, which will be the next step.\par

\begin{tabular}{l*{11}{c}r}
  \\
  \\
  && && Matrix C \\
  \\
  & &
  \textit{Column 0} &&
  \textit{Column 1} &&
  \textit{Column 2} \\
  \\
  \textit{Row 0} && AJ + \textbf{BM} + CP && AK + BN + \textbf{CQ} && \textbf{AL} + BO + CR \\
  \\
  \textit{Row 1} && DJ + EM + \textbf{FP} && \textbf{DK} + EN + FQ && DL + \textbf{EO} + FR \\
  \\
  \textit{Row 2} && \textbf{GJ} + HM + IP && GK + \textbf{HN} + IQ && GL + HO + \textbf{IR} \\
  \\
  \\
\end{tabular}

Each ape must now multiply their value of matrix A and their value of matrix B and then add the result to matrix C.  Matrix C now holds the sum of this and the previous multiplication (bolded below). \par

\begin{tabular}{l*{11}{c}r}
  \\
  \\
  && && Matrix C \\
  \\
  & &
  \textit{Column 0} &&
  \textit{Column 1} &&
  \textit{Column 2} \\
  \\
  \textit{Row 0} && \textbf{AJ + BM} + CP && AK + \textbf{BN + CQ} && \textbf{AL} + BO + \textbf{CR} \\
  \\
  \textit{Row 1} && DJ + \textbf{EM + FP} && \textbf{DK} + EN + \textbf{FQ} && \textbf{DL + EO} + FR \\
  \\
  \textit{Row 2} && \textbf{GJ} + HM + \textbf{IP} && \textbf{GK + HN} + IQ && GL + \textbf{HO + IR} \\
  \\
  \\
\end{tabular}

Now matrix C just needs the non-bolded values in the chart above.  We can copy the previous step, shifting matrix A to the left by 1 and shifting matrix B upwards by 1, in order to get the last pair of corresponding values from each matrix. \par

\begin{tabular}{l*{9}{c}r}
  \\
  \\
  & & & Matrix A \\
  \\
  & &
  \textit{Column 0} &
  \textit{Column 1} &
  \textit{Column 2} & \\
  \\
  \textit{Row 0} & & C & A & B && // Shift each value to the left once (circularly).\\
  \\
  \textit{Row 1} & & D & E & F \\
  \\
  \textit{Row 2} & & H & I & G \\
  \\
  \\
\end{tabular}

\begin{tabular}{l*{9}{c}r}
  \\
  \\
  & & & Matrix B \\
  \\
  & &
  \textit{Column 0} &
  \textit{Column 1} &
  \textit{Column 2} & \\
  \\
  \textit{Row 0} & & P & K & O && // Shift each value upwards once (circularly).\\
  \\
  \textit{Row 1} & & J & N & R \\
  \\
  \textit{Row 2} & & M & Q & L \\
  \\
  \\
\end{tabular}

Below are the bolded pairs that are now held by each ape. \par

\begin{tabular}{l*{11}{c}r}
  \\
  \\
  && && Matrix C \\
  \\
  & &
  \textit{Column 0} &&
  \textit{Column 1} &&
  \textit{Column 2} \\
  \\
  \textit{Row 0} && AJ + BM + \textbf{CP} && \textbf{AK} + BN + CQ && AL + \textbf{BO} + CR \\
  \\
  \textit{Row 1} && \textbf{DJ} + EM + FP && DK + \textbf{EN} + FQ && DL + EO + \textbf{FR} \\
  \\
  \textit{Row 2} && GJ + \textbf{HM} + IP && GK + HN + \textbf{IQ} && \textbf{GL} + HO + IR \\
  \\
  \\
\end{tabular}

We can see that the bolded pairs above are the same as the non-bolded pairs from the previous matrix C chart.  When these values are multiplied and added to matrix C, we will have finished multiplying matrix A and matrix B. \par
So, in order to complete a matrix multiplication with the ape network, there are 5 steps: \par
\begin{enumerate}
\item Number the apes into rows and columns.
\item Do the initial shifting of matrix A and matrix B.  Each row i of matrix A shift move to the left i times.  Each column j of matrix B must shift upwards j times.
\item Multiply the matrix A and matrix B values within each ape and add the result to a running total.
\item Shift matrix A to the left by one.  Shift matrix B upwards by one. \par
\item Repeat steps 3 and 4 N times, where N is equal to the length of the square matrices A and B.
\end{enumerate}
First let’s look at Step 1 (number the apes into rows and columns): \par

\begin{minted}{c}
  // Create variables in each ape for the ape’s row and column numbers.
// Set row and column to zero initially.
DeclareApeVar(row, Int);
DeclareApeVar(col, Int);
Set(row,IntConst(0));
Set(col,IntConst(0));

// We must number the row and column variables, because right now they are 
// all set to zero.
for (i = 0; i < N; i++){
// Using apeGet from the North will give us a zero in the top row of apes, since
// apeGet does not use a torus configuration.
eApeC(apeGet, row, row, getNorth);
Set(row, Add(row,IntConst(1)));

// Using apeGet from the West will give us a zero in the left-most column of
// apes, since apeGet does not use a torus configuration.
eApeC(apeGet, col, col, getWest);
Set(col, Add(col,IntConst(1)));
}

// We added one too many IntConst(1)s, because we still want the extra shift in the for loop.
// It’s easier to subtract IntConst(1) than to do another shift after the for loop.  Now we
// subtract IntConst(1).
Set(row, Sub(row,IntConst(1)));     // Sub() is a subtraction function found in scNova.h
Set(col, Sub(col,IntConst(1)));

// End Step 1.
\end{minted}

In order to check that the row and column numbers are correct, we can just add two lines of code: \par

\begin{minted}{c}
  TraceOneRegisterAllApes(row);
  TraceOneRegisterAllApes(col);
\end{minted}

These commands will print out the row and column numbers of each ape.  If everything worked correctly, we’d expect to see the row number for the first 8 apes equal 0; the row number for the next 8 apes equal 1, etc.  TraceOneRegisterAllApes() doesn’t print 0 values though, so the 0th row of all zeros won’t print.  Likewise, the 0th column of all zeros won’t print. \par
Now, let’s look at step 2 (the initial shifting of the matrices): \par

\begin{minted}{c}
  // Need to use Ape variables to manipulate matrices A and B.
DeclareApeVar(Aloaded, Approx);
Set(Aloaded, A); // Aloaded will hold the same values as matrix A.
DeclareApeVar(Bloaded, Approx);
Set(Bloaded, B); // Bloaded will hold the same values as matrix B.

// Shift each row i of matrix A to the left i times.  
// Shift each column j of matrix B upwards j times.
for (i = 0; i < N; i++){

// The Aloaded matrix is the same as matrix A.  Shift every Aloaded value to the left
// by one position. 
        emitGetTorus(Aloaded, getEast);

    // Check whether the row number of each ape is greater than i.
    // This masks the other rows.
        ApeIf(Gt(row, IntConst(i)));
        // If the row number is greater than i, then we want the matrix A value in that
        // row to shift to the left by one position.  This is easily done by setting the
        // non-masked values of matrix A to equal the corresponding values of
        // Aloaded, which we already shifted.
                 Set(A, Aloaded);
        ApeFi(); // Clear the masking.
        
    // The Bloaded matrix is the same as matrix B.  Shift every Bloaded value upwards by
    // one position.
        emitGetTorus(Bloaded, getSouth);

    // Check whether the column number of each ape is greater than i.
    // This will mask apes in smaller columns.
        ApeIf(Gt(col, IntConst(i)));
        // If the  column number is greater than i, then we want to shift the matrix B
        // values in that column upwards by one position.  We do this by setting the
        // non-masked values of matrix B to equal the corresponding values of
        // Bloaded, which we already shifted.
                 Set(B, Bloaded);
        ApeFi(); // Clear the masking from ApeIf().
        
} // Ends for(i=0; i < N; i++).
\end{minted}

In order to check that this shifting is happening in the way that we expect, we can add TraceOneRegisterAllApes(A); and TraceOneRegisterAllApes(B); inside the for loop.  Each time through, we will see the shifts in the positions of the A and B matrices.  \par
Printing out the entire A and B matrices each time through the for loop provides us with a lot of data to look at.  Huge blocks of data can be difficult to read, so I first wanted to just look at a small section of that data.  Before I looked at the entire matrices, I added TraceOneRegisterOneApe(A, 3, 5) and TraceOneRegisterOneApe(A, 3, 6).  This only printed out the matrix A value in the ape at coordinates (3,5) and (3,6).  I could’ve used any adjacent coordinates; I just wanted an easy way to see quickly whether the shifting was happening as I expected.  By seeing that the value in (3,6) was shifting over to (3,5), I verified that my code was working for at least some of matrix A.  I did the same for matrix B with coordinates that were one on top of the other, to make sure upwards shifting was happening.  Having confirmed that these small sections of the matrix were shifting properly, I then printed out the entire matrices with TraceOneRegisterAllApes(). \par
Next, let’s look at steps 3, 4, and 5.  Because step 5 is a loop of steps 3 and 4, it’s easier to look at these three steps together.  Step 3 is the multiplication of the matrix A and matrix B value within each ape, and step 4 is the shifting of the matrices by one position. \par

\begin{minted}{c}
  // emitGetTorus won’t allow us to get values directly from matrix A or matrix B.  In order to
// use emitGetTorus we need to use a level of misdirection and use the ape variables
// Aloaded and Bloaded.  First, we reset Aloaded and Bloaded to equal matrix A and matrix
// B.
Set(Aloaded, A);
Set(Bloaded, B);
 
// We want a variable in each ape that holds the running total of the matrix multiplication.
DeclareApeVar(runningTotal, Approx);
Set(runningTotal,ApproxConst(0)); // The running total is initially set to zero.

i = 0;

// Multiplies the Aloaded and Bloaded elements in each Ape and adds the result to the
// running total.  Then shifts Aloaded to the left and Bloaded upwards by one position. 
// Repeats this N times.  When this loop is finished, the running total will hold the result of
// the matrix multiplication.
do {
            // runningTotal = runningTotal + (Aloaded * Bloaded)
        Set(runningTotal, Add(runningTotal, Mul(Aloaded, Bloaded)));

        emitGetTorus(Aloaded, getEast); // Shifts Aloaded to the left.
        emitGetTorus(Bloaded, getSouth); // Shifts Bloaded upwards.

        i++;
} while(i < N);

// Resets matrix B to what it was before the multiplication, since we didn’t want this function
// to alter matrix B.
Set(B, Bsaved);
// Sets matrix A equal to the runningTotal.  Matrix A now will hold the result of matrix A * B.
Set(A, runningTotal);
\end{minted}

In order to check that steps 3, 4, and 5 are working as planned, we can add TraceOneRegisterOneApe and TraceOneRegisterAllApes functions (looking at the running total, Aloaded, and Bloaded variables) within the do-while loop.  This allows us to see the multiplication as it happens.  At the end of the multiplication, we can print out the entire A and B matrices, to make sure they’re equal to the multiplication result (A) and original values (B). \par
That completes the matrix multiplication function.  Below is the complete function: \par

\inputminted{c}{mm-emitMatrixMul.c}

Now we’ve created all the emit functions.  These functions still need to be loaded into a kernel and ran.  Our test function will do this.  Additionally, our test function will check to see whether our emitMatrixMul function is working. \par 
    First, the test function needs to load the instructions necessary to set our matrices A and B to the desired values.  Next, test will load the instruction to multiply the matrices.  Then, test will load the instruction to copy matrix A from the Apes back to the CU.  Test can then prepare to check matrix A against test’s calculation of what matrix A should be. Once all of these instructions have been loaded into the kernel, the kernel can be executed, and the Ape matrix multiplication will run. \par

    \inputminted{c}{mm-tests.c}

    Finally, the last function to look at is the check() function, which we use inside tests() in order to compare the correct matrix multiplication results against the results emitMatrixMul gave us.  If the difference is smaller than .02, we conclude that emitMatrixMul gave us the correct result.  Remember, emitMatrixMul won’t give us the exact result, because Apes use approximate variables rather than floats or ints. \par

    \inputminted{c}{mm-check.c}

    Now that we’ve looked at every section of the program, below is the full piece of code: \par

    \begin{minted}{c}
      /*********************************************************************************
    (c) Copyright 2011-2016 Singular Computing LLC

    This file and related materials are Confidential Information
    and Proprietary Property of Singular Computing LLC.

**********************************************************************************/


/*

  Simple matrix multiplication example code for S1 using Nova.

  To compile: 
  gcc -lm -o matrixMultiplication matrixMultiplication.c scNova.c scAcceleratorAPI.c scEmulator.c scArithmetic178.c pmbus.c


  We keep a matrix (A) stored persistently in the S1,
  and multiply it with a temporary matrix (B), storing the result in A.

  For this code, A and B have fixed, identical, square NxN shapes, where N=8.
  Matrix elements are stored one per core.

*/


#include <stdio.h>
#include <stdlib.h>
#include <string.h>
#include <math.h>
#include <sys/types.h>
#include <sys/stat.h>
#include <time.h>

// Include the Singular Software libraries.
#include "scAcceleratorAPI.h"
#include "scNova.h"


int emulated; // 1 for emulated machine.  0 for real S1 hardware.

// If emulated, scTotalCyclesTake is the total number of cycles
// taken to run the last kernel.
extern int scTotalCyclesTaken;

// If emulated, *scEmulatedMachineState is a pointer to the
// emulated machine's state.
extern MachineState *scEmulatedMachineState;  

extern LLKernel *llKernel;

// Variables for the size info of the machine.
int chipRows;
int chipCols;
int apeRows;
int apeCols;

int traceFlags;

// Define the length of the square matrices.
#define N 8

// Declare space for matrices A and B on the CPU, in float format.
float floatA[N][N];
float floatB[N][N];

// Declare space for a matrix on the CPU, in approx format (16 bits),
// aligned at 64 bits.
uint64_t approxM[(N*N)/4];

// Declare often used Nova Constants.
Declare(a0);
Declare(a1);

// Declare names of A and B matrices in Ape memory.
Declare(A);
Declare(B);

void defineNames () {
// Initialization routine to define the names above.
    a0 = AConst(0);
    a1 = AConst(1);
    ApeMem(A, Approx);
    ApeMem(B, Approx);
}

    \end{minted}
    \inputminted{c}{mm-emitCopyMatrixFromCUToApes.c}
    \inputminted{c}{mm-emitCopyMatrixFromApesToCU.c}
    \inputminted{c}{mm-copyBFromCU.c}
    \inputminted{c}{mm-copyAFromCU.c}

    \begin{minted}{c}
void emitMatrixSet () {
    Set(A, B);
}
\end{minted}

    \inputminted{c}{mm-emitGetTorus.c}
    \inputminted{c}{mm-emitMatrixMul.c}
\inputminted{c}{mm-check.c}
\inputminted{c}{mm-tests.c}
\inputminted{c}{mm-main.c}

    TODO: \begin{enumerate}
    \item Load the matrix with more interesting numbers than all zeros.
    \item Put in information about TraceFlags.
    \item Ask Joe about what DDR and randomize do in initmachine.
    \item Ask Joe - go through emitCopyMatrixFrom and To code.
    \item Ask Joe - emitGetTorus not fully understood.
    \end{enumerate}

\end{document}
